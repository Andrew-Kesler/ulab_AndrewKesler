\documentclass{article}
\usepackage{graphicx} % Required for inserting images
\usepackage[letterpaper, portrait,margin=1in]{geometry} 
\usepackage{float}
\usepackage{amsmath} 
\usepackage{physics} 
\usepackage{hyperref} 
\usepackage{parskip}

   
\title{Doctor, is this normal?\\ULAB, Division of Sincerely Serious Science}
\author{Andrew Kesler}
\date{\today}
\begin{document} 


\maketitle
\section{1 Introduction}
According to my sources, around May 2021 an image began circulating around the internet about broken sidewalk. The image, which can be seen in\autoref{Figure 1:}, claims that the sidewalk crack resembles the form of a normal distribution. This lab report will test that claim by fitting the curve with a normal standard distribution (also called a Gaussian distribution). 
\begin{figure}[h]
        \centering
        \includegraphics[width=0.8\linewidth]{img.jpg}
        \caption{Sidewalk image from the internet.}
        \label{Figure 1:}
        \end{figure}
\section{2 Methods}
\subsection{2.1 Preprocessing}
First we used in Perspective Wrap tool in Photoshop to correct the titled perspective of the image. Then, we worked with the Python Imaging Library and NumPy [1] and converted our image to grayscale. 

A threshold mask was applied to the image so that the only  dark pixels would be retained. Note that some minimal manual processing was required to remove the vertical sidewalk crack as well as the outlier pixels.

Finally, the location of the dark pixels are recorded as coordinates in a Cartesian plane. 
\subsection{2.2 Curve Fitting}
\begin{equation}
f(x) = a \cdot e^{- \frac{1}{2} \frac{(x - \mu)^2}{\sigma}}
\end{equation}
\section{3. Results}
After our data analysis, \autoref{Figure 2:} reveals that a normal distributions fits our data well. 
\begin{figure}[h]
    \centering
    \includegraphics[width=.8\textwidth]{fit.png}
    \caption{Top: Original image, Middle: processed image with perspective correction and greyscale transformation, Bottom: curve fit of the sidewalk crack points.}
    \label{Figure 2:}
\end{figure}    

Furthermore, we determined the fit to have a reduced chi-squatted value,
\begin{equation} 
\chi^2_\nu = \frac{1}{\nu} \cdot \sum \frac{(O - E)^2}{E} = 0.22
\end{equation}
where v, is he number of degrees of freedom. Of note, we used the Pearson chi-squared definition because it is difficult to estimate the uncertainty of our points.
\bibliographystyle{ieeetr}
\bibliography{ref}
\nocite{*}
\end{document}